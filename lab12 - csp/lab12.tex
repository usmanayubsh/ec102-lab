\documentclass{beamer}
\usepackage[utf8]{inputenc}

\usepackage{amsmath}
\usepackage{graphicx}
\usepackage{url}
\usepackage{fancyvrb}
\usepackage{xcolor}

\usetheme{Madrid}
\usecolortheme{seahorse}

\usepackage{inconsolata}
\usepackage[scaled]{helvet}
\renewcommand*\familydefault{\sfdefault}
\usepackage[T1]{fontenc}

\usepackage{listings}
\usepackage{color}

\definecolor{codegreen}{rgb}{0,0.6,0}
\definecolor{codegray}{rgb}{0.5,0.5,0.5}
\definecolor{codepurple}{rgb}{0.58,0,0.82}
\definecolor{backcolour}{rgb}{0.95,0.95,0.92}

\mode<presentation>

\definecolor{orange}{HTML}{BC2E07}

\usepackage{hyperref}
\hypersetup{
    colorlinks,
    linkcolor=orange,
    urlcolor=blue
}

\lstdefinestyle{mystyle}{
    language=C++,
    basicstyle=\ttfamily\footnotesize,
    backgroundcolor=\color{backcolour},
    commentstyle=\color{codegreen},
    keywordstyle=\color{magenta},
    numberstyle=\tiny\color{codegray},
    stringstyle=\color{codepurple},
    breakatwhitespace=false,
    breaklines=true,
    captionpos=b,
    keepspaces=true,
    numbers=left,
    numbersep=5pt,
    showspaces=false,
    showstringspaces=false,
    showtabs=false,
    tabsize=2
}

\title{Lab \# 11: Functions - I}
\subtitle{EC-102 -- Computer Systems and Programming}

\author{Usman Ayub Sheikh}
\institute{School of Mechanical and Manufacturing Engineering (SMME), \\ National University of Sciences and Technology (NUST)}
\date{\today}

\begin{document}
\begin{frame}
    \titlepage
\end{frame}

\begin{frame}
    \frametitle{Outline}
        \tableofcontents
\end{frame}

\begin{frame}
    \frametitle{What are Functions?}
    \section{Introduction to Functions} % (fold)
    \label{sec:functions}
    \subsection{What?} % (fold)
    \label{sub:what}
    \begin{itemize}
        \item Groups a number of program statements into a unit and gives it a name
        \item This unit can then be invoked from other parts of the program
        \item So far, we have studied only one function i.e. \texttt{main()} function
    \end{itemize}
\end{frame}

\begin{frame}
    \frametitle{Why Use Functions?}
    \subsection{Why?} % (fold)
    \label{sub:why}
    \begin{itemize}
        \item To aid in the conceptual organization of a program
        \item To reduce program size
        \item The function's code is stored in only one place in memory
    \end{itemize}
\end{frame}

\begin{frame} [fragile]
    \frametitle{Why Use Functions?}
    Write a C++ program that displays the following table to the console
    \lstset{style=mystyle}
\begin{lstlisting}
*********************************************
Data-type  Range
*********************************************
char       -128 to 127
short      -32,768 to 32,767
int        System dependent
long       -2,147,483,648 to 2,147,483,647
*********************************************
\end{lstlisting}
\end{frame}

\begin{frame}[fragile]
    \frametitle{A Simple Function}
    \subsection{How?} % (fold)
    \label{sub:how}
    \subsubsection{A Simple Function} % (fold)
    \label{ssub:a_simple_function}
    \lstset{style=mystyle}
\begin{lstlisting}
// demonstrates a simple function
#include <iostream>
using namespace std;

void starline(); // function declaration

int main()
{
    starline();
    cout << "Data-type  Range" << endl;
    starline(); // call to function

    cout << "char    -128 to 127" << endl;
    cout << "short   -32,768 to 32,767" << endl;
    cout << "int     System dependent" << endl;
    cout << "long    -2,147,483,648 to 2,147,483,647" << endl;
    starline();

    return 0;
}
\end{lstlisting}
\end{frame}

\begin{frame}[fragile]
    \frametitle{A Simple Function}
    \lstset{style=mystyle}
\begin{lstlisting} [firstnumber=21]
// function definition
void starline() // function declarator
{               // function body starts here
    for(int j = 0; j < 45; j++)
    {
        cout << "*";
    }
    cout << endl;
}
\end{lstlisting}
\end{frame}

\begin{frame}[fragile]
    \frametitle{The Function Declaration}
    \subsubsection{The Function Declaration} % (fold)
    \label{ssub:the_function_declaration}
    \begin{itemize}
        \item Also known as a prototype
        \item Just as a variable can't be used without telling the compiler what it is, functions also need to be declared before they are called
        \lstset{style=mystyle}
\begin{lstlisting}
void starline();
\end{lstlisting}
    \item Tells the compiler that at some later point we plan to present a function called \texttt{starline}
    \item The keyword \texttt{void} specifies that the function has no return value and the empty parentheses indicate that it takes no arguments
    \item Is terminated by a semicolon
    \item The information in the declaration is also sometimes referred to as the function \textbf{signature}
    \end{itemize}
\end{frame}

\begin{frame}[fragile]
    \frametitle{Calling the Function}
    \subsubsection{Calling the Function} % (fold)
    \label{ssub:calling_the_function}
    \begin{itemize}
        \item The function is called/invoked as follows:
        \lstset{style=mystyle}
\begin{lstlisting}
starline();
\end{lstlisting}
    \item The function name followed by parentheses
    \item The syntax is very similar to that of the declaration except that the return type is not used
    \item Terminated by a semicolon
    \item Executing the call statement causes the function to be executed i.e. control is transferred to the function
    \end{itemize}
\end{frame}

\begin{frame}[fragile]
    \frametitle{The Function Definition}
    \subsubsection{The Function Definition} % (fold)
    \label{ssub:the_function_definition}
    \begin{itemize}
        \item The definition contains the \textbf{actual code} for the function
        \item Here goes the definition for \texttt{starline()}:
        \lstset{style=mystyle}
\begin{lstlisting}
void starline()
{
    for(int j = 0; j < 45; j++)
    {
        cout << "*";
    }
    cout << endl;
}
\end{lstlisting}
        \item The definition consists of a line called the \textbf{declarator}, followed by the function \textbf{body}
        \item The function body is composed of the statements that make up the function and is delimited by braces
        \item The declarator must agree with the declaration and is \textbf{not} terminated by a semicolon
    \end{itemize}
\end{frame}

\begin{frame} [fragile]
    \frametitle{Comparison with Library Functions}
    \section{Comparison with Library Functions} % (fold)
    \label{sec:comparison_with_library_functions}
    \begin{itemize}
        \item For a library function, we don't need to write the declaration or definition
        \lstset{style=mystyle}
\begin{lstlisting}
pow(2, 4);
\end{lstlisting}
        \item The declaration is in the header file specified at the beginning of the program e.g. \texttt{cmath} for \texttt{pow} function
        \item The definition is in a library file that's linked automatically to our program when we build it
    \end{itemize}
\end{frame}

\begin{frame}
    \frametitle{Passing Arguments to Functions}
    \section{Passing Arguments to Functions} % (fold)
    \label{sec:passing_arguments_to_functions}
    \begin{itemize}
        \item An argument is a piece of data passed from a program to the function
        \item Arguments allow a function to operate with different values
        \item By using arguments, we can create a function that prints any character any number of times
    \end{itemize}
\end{frame}

\begin{frame} [fragile]
    \frametitle{Passing Constants as Arguments}
    \subsection{Passing Constants} % (fold)
    \label{sub:passing_constants}
    \lstset{style=mystyle}
\begin{lstlisting}
// demonstrates function arguments
#include <iostream>
using namespace std;

void rep_char(char, int); // function declaration

int main()
{
    rep_char('-', 45);
    cout << "Data-type  Range" << endl;
    rep_char('-', 45); // call to function

    cout << "char    -128 to 127" << endl;
    cout << "short   -32,768 to 32,767" << endl;
    cout << "int     System dependent" << endl;
    cout << "long    -2,147,483,648 to 2,147,483,647" << endl;
    rep_char('=', 45);

    return 0;
}
\end{lstlisting}
\end{frame}

\begin{frame} [fragile]
    \frametitle{Passing Constants as Arguments}
    \lstset{style=mystyle}
\begin{lstlisting} [firstnumber=21]
void rep_char(char ch, int n)
{
    for(int j = 0; j < n; j++)
    {
        cout << ch;
    }
    cout << endl;
}
\end{lstlisting}
\end{frame}

\begin{frame} [fragile]
    \frametitle{Passing Variables as Arguments}
    \subsection{Passing Variables} % (fold)
    \label{sub:passing_variables}
    \lstset{style=mystyle}
\begin{lstlisting}
// demonstrates function arguments
#include <iostream>
using namespace std;

void rep_char(char, int); // function declaration

int main()
{
    char chin;
    int nin;

    cout << "Enter a character: ";
    cin >> chin;
    cout << "Enter number of times to repeat it: ";
    cin >> nin;

    rep_char(chin, nin);

    return 0;
}
\end{lstlisting}
\end{frame}

\begin{frame} [fragile]
    \frametitle{Passing Variables as Arguments}
    \lstset{style=mystyle}
\begin{lstlisting} [firstnumber=21]
void rep_char(char ch, int n)
{
    for(int j = 0; j < n; j++)
    {
        cout << ch;
    }
    cout << endl;
}
\end{lstlisting}
\end{frame}

\begin{frame} [fragile]
    \frametitle{Passing Structures as Arguments}
    \subsection{Passing Structures} % (fold)
    \label{sub:passing_structures}
    \lstset{style=mystyle}
\begin{lstlisting}
// demonstrates passing structure as argument
#include <iostream>
using namespace std;
struct Distance
{
    int feet;
    float inches;
};

void engldisp(Distance);

int main()
{
    Distance d1, d2;
    cout << "Enter feet: "; cin >> d1.feet;
    cout << "Enter inches: "; cin >> d1.inches;
    cout << "\nEnter feet: ";
    cin >> d2.feet;
    cout << "Enter inches: ";
    cin >> d2.inches;
\end{lstlisting}
\end{frame}

\begin{frame} [fragile]
    \frametitle{Passing Structures as Arguments}
    \lstset{style=mystyle}
\begin{lstlisting} [firstnumber=21]
    cout << "\nd1 = ";
    engldisp(d1);
    cout << "\nd2 = ";
    engldisp(d2);
    cout << endl;
    return 0;
}

void engldisp(Distance dd)
{
    cout << dd.feet << "\'-" << dd.inches << "\"";
}
\end{lstlisting}
\end{frame}

\end{document}