\documentclass{beamer}
\usepackage[utf8]{inputenc}

\usepackage{amsmath}
\usepackage{graphicx}
\usepackage{url}
\usepackage{fancyvrb}
\usepackage{xcolor}

\usetheme{Madrid}
\usecolortheme{seahorse}

\usepackage{inconsolata}
\usepackage[scaled]{helvet}
\renewcommand*\familydefault{\sfdefault}
\usepackage[T1]{fontenc}

\usepackage{listings}
\usepackage{color}

\definecolor{codegreen}{rgb}{0,0.6,0}
\definecolor{codegray}{rgb}{0.5,0.5,0.5}
\definecolor{codepurple}{rgb}{0.58,0,0.82}
\definecolor{backcolour}{rgb}{0.95,0.95,0.92}

\mode<presentation>

\definecolor{orange}{HTML}{BC2E07}

\usepackage{hyperref}
\hypersetup{
    colorlinks,
    linkcolor=orange,
    urlcolor=blue
}

\lstdefinestyle{mystyle}{
    language=C++,
    basicstyle=\ttfamily\footnotesize,
    backgroundcolor=\color{backcolour},
    commentstyle=\color{codegreen},
    keywordstyle=\color{magenta},
    numberstyle=\tiny\color{codegray},
    stringstyle=\color{codepurple},
    breakatwhitespace=false,
    breaklines=true,
    captionpos=b,
    keepspaces=true,
    numbers=left,
    numbersep=5pt,
    showspaces=false,
    showstringspaces=false,
    showtabs=false,
    tabsize=2
}

\title{Lab \# 12: Functions - II}
\subtitle{EC-102 -- Computer Systems and Programming}

\author{Usman Ayub Sheikh}
\institute{School of Mechanical and Manufacturing Engineering (SMME), \\ National University of Sciences and Technology (NUST)}
\date{\today}

\begin{document}

\begin{frame}
    \titlepage
\end{frame}

\begin{frame}
    \frametitle{Outline}
        \tableofcontents
\end{frame}

\begin{frame}
    \frametitle{Returning Values from Functions}
    \section{Returning Values from Functions} % (fold)
    \label{sec:returning_values}
    \begin{itemize}
        \item When a function completes its execution, it can return a single value to the calling program
        \item Usually this value consists of an answer to the problem the function has solved
    \end{itemize}
\end{frame}

\begin{frame} [fragile]
    \frametitle{Returning Values from Functions}
    \lstset{style=mystyle}
\begin{lstlisting}
// demonstrates return values, converts pounds to kg
#include <iostream>
using namespace std;

float lbstokgs(float); // declaration

int main()
{
    float lbs, kgs;
    cout << "Enter your weight in pounds: "; cin >> lbs;
    kgs = lbstokgs(lbs);
    cout << "Your weight in kilograms is: " << kgs << endl;
    return 0;
}

float lbstokgs(float pounds)
{
    float kilograms = 0.453592 * pounds;
    return kilograms;
}
\end{lstlisting}
\end{frame}

\begin{frame}
    \frametitle{Returning Values from Functions}
    \begin{itemize}
        \item When a function returns a value, the data type of this value must be specified
        \item In the declaration \texttt{float lbstokgs(float);}, the first \texttt{float} represents the return type
        \item When a function returns a value, the call to the function \texttt{lbstokgs(lbs)} is considered to take on the value returned by the function
    \end{itemize}
\end{frame}

\begin{frame}
    \frametitle{The \texttt{return} Statement}
    \subsection{The \texttt{return} Statement} % (fold)
    \label{sub:the_return_statement}
    \begin{itemize}
        \item While many arguments may be sent to a function, only one argument may be returned from it
        \item Always include a function's return type in the function declaration. If it does not return anything, use the keyword \texttt{void} to indicate this
    \end{itemize}
\end{frame}

\begin{frame} [fragile]
    \frametitle{Eliminating Unnecessary Variables}
    \subsection{Eliminating Unnecessary Variables} % (fold)
    \label{sub:eliminating_unnecessary_variables}
    \lstset{style=mystyle}
\begin{lstlisting}
// eliminates unnecessary variables
#include <iostream>
using namespace std;

float lbstokgs(float); // declaration

int main()
{
    float lbs;
    cout << "Enter your weight in lbs: "; cin >> lbs;
    cout << "Your weight in kgs is: " << lbstokgs(lbs) << endl;
    return 0;
}

float lbstokgs(float pounds)
{
    return 0.453592 * pounds;
}
\end{lstlisting}
\end{frame}

\begin{frame} [fragile]
    \frametitle{Returning Structure Variables}
    \section{Returning Structure Variables} % (fold)
    \label{sec:returning_structure_variables}
    \lstset{style=mystyle}
\begin{lstlisting}
// demonstrates returning a structure
#include <iostream>
using namespace std;

struct Distance
{
    int feet;
    float inches;
};

Distance addengl(Distance, Distance);
void engldisp(Distance);

int main()
{
    Distance d1, d2, d3;
    cout << "\nEnter feet: "; cin >> d1.feet;
    cout << "Enter inches: "; cin >> d1.inches;
    cout << "\nEnter feet: "; cin >> d2.feet;
\end{lstlisting}
\end{frame}

\begin{frame} [fragile]
    \frametitle{Returning Structure Variables}
    \lstset{style=mystyle}
\begin{lstlisting} [firstnumber=20]
    cout << "Enter inches: "; cin >> d2.inches;
    d3 = addengl(d1, d2);
    cout << endl;

    cout << "Sum of ";
    engldisp(d1); cout << " and ";
    engldisp(d2); cout << " is: ";
    engldisp(d3); cout << endl;
    return 0;
}
\end{lstlisting}
\end{frame}

\begin{frame} [fragile]
    \frametitle{Returning Structure Variables}
    \lstset{style=mystyle}
\begin{lstlisting} [firstnumber=30]
Distance addengl(Distance dd1, Distance dd2)
{
    Distance dd3;
    dd3.inches = dd1.inches + dd2.inches; //add the inches
    dd3.feet = 0;
    if(dd3.inches >= 12.0)
    {
        dd3.inches -= 12.0;
        dd3.feet++;
    }
    dd3.feet += dd1.feet + dd2.feet;
    return dd3;
}
\end{lstlisting}
\end{frame}

\begin{frame} [fragile]
    \frametitle{Returning Structure Variables}
    \lstset{style=mystyle}
\begin{lstlisting} [firstnumber=43]
void engldisp(Distance dd)
{
    cout << dd.feet << "\'-" << dd.inches << "\"";
}
\end{lstlisting}
\end{frame}

\begin{frame}
    \frametitle{Exercise 1}
    \section{Exercises} % (fold)
    \label{sec:exercises}
    \subsection{Exercise 1} % (fold)
    \label{sub:exercise_1}
    \begin{itemize}
        \item Write a function that
        \begin{itemize}
            \item takes two \texttt{Distance} values as arguments, and
            \item returns the larger one.
        \end{itemize}
        \item Include a \texttt{main()} program that
        \begin{itemize}
            \item accepts two \texttt{Distance} values from the user,
            \item compares them, and
            \item displays the larger.
        \end{itemize}
    \end{itemize}
\end{frame}

\begin{frame}
    \frametitle{Exercise 2}
    \subsection{Exercise 2} % (fold)
    \label{sub:exercise_2}
    \begin{itemize}
        \item Write a function called \texttt{hms\_to\_secs()} that
        \begin{itemize}
            \item takes three \texttt{int} values -- for hours, minutes, and seconds -- as arguments, and
            \item returns the equivalent time in seconds (type \texttt{long}).
        \end{itemize}
        \item Create a program that
        \begin{itemize}
            \item exercises this function by repeatedly obtaining a time value in hours, minutes, and seconds from the user (format 12:59:59),
            \item calling the function, and
            \item displaying the value of seconds it returns.
        \end{itemize}
    \end{itemize}
\end{frame}

\begin{frame}
    \frametitle{Exercise 3}
    \subsection{Exercise 3} % (fold)
    \label{sub:exercise_3}
    \begin{itemize}
        \item Create a structure called \texttt{Time}. Its three members, all of type \texttt{int}, should be called \texttt{hours}, \texttt{minutes}, and \texttt{seconds}.
        \item Create two functions,
        \begin{itemize}
            \item One of them, \texttt{time\_to\_secs()}, should take as its only argument a structure of type time, and return the equivalent in seconds (type \texttt{long})
            \item The other one, \texttt{secs\_to\_time()} should take as its only argument a time in seconds (type \texttt{long}), and return a structure of type \texttt{time}
        \end{itemize}
        \item Write a program that
        \begin{itemize}
            \item exercises these functions by obtaining two time values from the user in hh:mm:ss format, and
            \item printing out the sum of them in hh:mm:ss format
        \end{itemize}
    \end{itemize}
\end{frame}

\end{document}