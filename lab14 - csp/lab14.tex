\documentclass{beamer}
\usepackage[utf8]{inputenc}

\usepackage{amsmath}
\usepackage{graphicx}
\usepackage{url}
\usepackage{fancyvrb}
\usepackage{xcolor}
\usepackage{adjustbox}

\usetheme{Madrid}
\usecolortheme{seahorse}

\usepackage{inconsolata}
\usepackage[scaled]{helvet}
\renewcommand*\familydefault{\sfdefault}
\usepackage[T1]{fontenc}

\usepackage{listings}
\usepackage{color}

\definecolor{codegreen}{rgb}{0,0.6,0}
\definecolor{codegray}{rgb}{0.5,0.5,0.5}
\definecolor{codepurple}{rgb}{0.58,0,0.82}
\definecolor{backcolour}{rgb}{0.95,0.95,0.92}

\mode<presentation>

\definecolor{orange}{HTML}{BC2E07}

\usepackage{hyperref}
\hypersetup{
    colorlinks,
    linkcolor=orange,
    urlcolor=blue
}

\lstdefinestyle{mystyle}{
    language=C++,
    basicstyle=\ttfamily\footnotesize,
    backgroundcolor=\color{backcolour},
    commentstyle=\color{codegreen},
    keywordstyle=\color{magenta},
    numberstyle=\tiny\color{codegray},
    stringstyle=\color{codepurple},
    breakatwhitespace=false,
    breaklines=true,
    captionpos=b,
    keepspaces=true,
    numbers=left,
    numbersep=5pt,
    showspaces=false,
    showstringspaces=false,
    showtabs=false,
    tabsize=2
}

\usepackage{datetime}
\newdate{date}{20}{11}{2015}

\title{Lab \# 13: Functions - III}
\subtitle{EC-102 -- Computer Systems and Programming}

\author{Usman Ayub Sheikh}
\institute{School of Mechanical and Manufacturing Engineering (SMME), \\ National University of Sciences and Technology (NUST)}
\date{\displaydate{date}}

\begin{document}

\begin{frame}
    \titlepage
\end{frame}

\begin{frame}
    \frametitle{Outline}
        \tableofcontents
\end{frame}

\begin{frame}
    \frametitle{Reference Arguments}
    \section{Reference Arguments} % (fold)
    \label{sec:reference_arguments}
    \subsection{Definition} % (fold)
    \label{sub:def}
    \begin{itemize}
        \item An alias, a different name for a variable
        \item One of the most important uses is in passing arguments to functions
    \end{itemize}
\end{frame}

\begin{frame}
    \frametitle{Passing by Value vs Passing by Reference}
    \subsection{Passing by Value vs Passing by Reference} % (fold)
    \label{subsec:passing_by_ref}
    \begin{itemize}
        \item When arguments are passed by value, the called function creates a new variable of the same type as the argument and assigns the argument's value to it
        \item The function cannot access the original variable in the calling program, only the copy it created
        \item It is useful when the function does not need to modify the original variable in the calling program
        \item It offers insurance that the function cannot harm the original value
    \end{itemize}
\end{frame}

\begin{frame}
    \frametitle{Passing by Value vs Passing by Reference}
    \begin{itemize}
        \item Passing arguments by reference uses a different mechanism
        \item Instead of a value being passed to a function, a reference to the original variable, in the calling program, is passed
        \item An important advantage is that the function can access the actual variables in the calling program
        \item Provides a mechanism for passing more than one value from the function back to the calling program
    \end{itemize}
\end{frame}

\begin{frame} [fragile]
    \frametitle{Passing Simple Data Types by Reference}
    \section{Solved Example} % (fold)
    \label{sec:solved_example}
    \lstset{style=mystyle}
\begin{lstlisting}
// ref.cpp
// demonstrates passing by reference
#include <iostream>
using namespace std;
void intfrac(float, int&, float&); //declaration

int main()
{
    float number, fracpart;
    int intpart;

    do {
        cout << "\nEnter a real number: "; //number from user
        cin >> number;
        intfrac(number, intpart, fracpart); //find int and frac
        cout << "Integer part is " << intpart //print them
        << ", fraction part is " << fracpart << endl;
    } while( number != 0.0 ); //exit loop on 0.0
    return 0;
}
\end{lstlisting}
\end{frame}

\begin{frame} [fragile]
    \frametitle{Passing Simple Data Types by Reference}
    \lstset{style=mystyle}
\begin{lstlisting} [firstnumber=21]
// intfrac()
// finds integer and fractional parts of real number
void intfrac(float n, int& intp, float& fracp)
{
    intp = n;
    fracp = n - intp; //subtract integer part
}
\end{lstlisting}
\end{frame}

\begin{frame}
    \frametitle{Exercise}
    \section{Exercise} % (fold)
    \label{sec:exercise}
    Write a function that
    \begin{itemize}
        \item takes a reference to a binary number (base 2) as an argument, and
        \item converts that number to a decimal number (base 10).
    \end{itemize}
    Write a program that
        \begin{itemize}
            \item exercises this function by obtaining a binary number from the user,
            \item and printing out the equivalent decimal number.
        \end{itemize}
    % section exercise (end)
\end{frame}

\end{document}