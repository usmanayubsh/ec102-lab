\documentclass{beamer}
\usepackage[utf8]{inputenc}

\usepackage{amsmath}
\usepackage{graphicx}
\usepackage{url}
\usepackage{fancyvrb}
\usepackage{xcolor}
\usepackage{adjustbox}

\usetheme{Madrid}
\usecolortheme{seahorse}

\usepackage{inconsolata}
\usepackage[scaled]{helvet}
\renewcommand*\familydefault{\sfdefault}
\usepackage[T1]{fontenc}

\usepackage{listings}
\usepackage{color}

\definecolor{codegreen}{rgb}{0,0.6,0}
\definecolor{codegray}{rgb}{0.5,0.5,0.5}
\definecolor{codepurple}{rgb}{0.58,0,0.82}
\definecolor{backcolour}{rgb}{0.95,0.95,0.92}

\mode<presentation>

\definecolor{orange}{HTML}{BC2E07}

\usepackage{hyperref}
\hypersetup{
    colorlinks,
    linkcolor=orange,
    urlcolor=blue
}

\lstdefinestyle{mystyle}{
    language=C++,
    basicstyle=\ttfamily\footnotesize,
    backgroundcolor=\color{backcolour},
    commentstyle=\color{codegreen},
    keywordstyle=\color{magenta},
    numberstyle=\tiny\color{codegray},
    stringstyle=\color{codepurple},
    breakatwhitespace=false,
    breaklines=true,
    captionpos=b,
    keepspaces=true,
    numbers=left,
    numbersep=5pt,
    showspaces=false,
    showstringspaces=false,
    showtabs=false,
    tabsize=2
}

\usepackage{datetime}
\newdate{date}{28}{12}{2015}

\title{Lab \# 16: Arrays and Strings -- Part 3}
\subtitle{EC-102 -- Computer Systems and Programming}

\author{Usman Ayub Sheikh}
\institute{School of Mechanical and Manufacturing Engineering (SMME), \\ National University of Sciences and Technology (NUST)}
\date{\displaydate{date}}

\begin{document}

\begin{frame}
    \titlepage
\end{frame}

\begin{frame}
    \frametitle{Outline}
        \tableofcontents
\end{frame}

\begin{frame}[fragile]\frametitle{Copying a String -- The Hard Way}
\section{Copying a String} % (fold)
\label{sec:copying_a_string}
    \subsection{The Hard Way} % (fold)
    \label{sub:the_hard_way}
    \lstset{style=mystyle}
\begin{lstlisting}
#include<iostream>
#include<cstring>
using namespace std;

int main()
{
  char str1[] = "Oh, Captain, my Captain! "
    "our fearful trip is done.";

  const int MAX = 80;
  char str2[MAX];

  for(int j = 0; j < strlen(str1); j++)
    str2[j] = str1[j];

  str2[-1] = '\0';
  cout << str2 << endl;
  return 0;
}\end{lstlisting}
    % subsection the_hard_way (end)
% section copying_a_string (end)
\end{frame}

\begin{frame}\frametitle{Copying a String -- The Hard Way}
    \begin{itemize}
        \item A string constant \texttt{str1} and a string variable \texttt{str2}
        \item A \texttt{for} loop to copy the string constant to the string variable
        \item A cstring library function \texttt{strlen} used to find the length of a C-string
        \item The copied version of the string must be terminated with a null
        \begin{center}
            \texttt{str2[j] = `\textbackslash0';}
        \end{center}
    \end{itemize}
\end{frame}

\begin{frame}[fragile]\frametitle{Copying a String -- The Easy Way}
    \subsection{The Easy Way} % (fold)
    \label{sub:the_easy_way}
    \lstset{style=mystyle}
\begin{lstlisting}
#include<iostream>
#include<cstring>
using namespace std;

int main()
{
  char str1[] = "Tiger, tiger, burning bright\n"
    "In the forests of the night.";

  const int MAX = 80;
  char str2[MAX];

    strcpy(str2, str1);

  cout << str2 << endl;
  return 0;
}
\end{lstlisting}
    Note that we are calling \texttt{strcpy} function with \textbf{destination first}.
    % subsection the_easy_way (end)
\end{frame}

\begin{frame}[fragile]\frametitle{Arrays of Strings}
    \section{Arrays of Strings} % (fold)
    \label{sec:arrays_of_strings}
    \lstset{style=mystyle}
\begin{lstlisting}
#include<iostream>
#include<cstring>
using namespace std;

int main()
{
    const int DAYS = 7;
    const int MAX = 10;

    char star[DAYS][MAX] = {"Sunday", "Monday", "Tuesday",
    "Wednesday", "Thursday", "Friday", "Saturday"};

    for(int j = 0; j < DAYS; j++)
        cout << star[j] << endl;
    return 0;
}
\end{lstlisting}
    % section arrays_of_strings (end)
\end{frame}

\begin{frame}\frametitle{Exercise 1}
    \section{Exercises} % (fold)
    \label{sec:exercises}
    \subsection{Exercise 1} % (fold)
    \label{sub:exercise_1}
    Start with a program that allows the user to input a number of integers, and then stores them in an int array. \\
    Write a function called \texttt{maxint()} that:
    \begin{itemize}
        \item Goes through the array, element by element, looking for the largest one.
        \item The function should take as arguments the array and the number of elements in it, and
        \item Return the index number of the largest element.
    \end{itemize}
    The program should call this function and then display the largest element and its index number.
    % subsection exercise (end)
    % section exercises (end)
\end{frame}

\begin{frame}\frametitle{Exercise 2}
    \subsection{Exercise 2} % (fold)
    \label{sub:exercise_2}
    Write a function called \texttt{reversit()} that reverses a C-string (an array of char).
    \begin{itemize}
        \item Use a for-loop that swaps the first and last characters, then the second and next-to-last characters, and so on.
        \item The string should be passed to \texttt{reversit()} as an argument.
        \item Write a program to exercise \texttt{reversit()}. The program should get a string from the user, call \texttt{reversit()}, and print out the result.
    \end{itemize}
    Use an input method that allows embedded blanks. Test the program with Napoleon’s famous phrase,  ``Able was I ere I saw Elba.''
       % subsection exercise (end)
    % section exercises (end)
\end{frame}
\end{document}