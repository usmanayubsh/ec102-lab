\documentclass{beamer}
\usepackage[utf8]{inputenc}

\usepackage{amsmath}
\usepackage{graphicx}
\usepackage{url}
\usepackage{fancyvrb}
\usepackage{xcolor}

\usetheme{Madrid}
\usecolortheme{seahorse}

\usepackage{inconsolata}
\usepackage[scaled]{helvet}
\renewcommand*\familydefault{\sfdefault}
\usepackage[T1]{fontenc}

\usepackage{listings}
\usepackage{color}

\definecolor{codegreen}{rgb}{0,0.6,0}
\definecolor{codegray}{rgb}{0.5,0.5,0.5}
\definecolor{codepurple}{rgb}{0.58,0,0.82}
\definecolor{backcolour}{rgb}{0.95,0.95,0.92}

\mode<presentation>

\definecolor{orange}{HTML}{BC2E07}

\usepackage{hyperref}
\hypersetup{
    colorlinks,
    linkcolor=orange,
    urlcolor=blue
}

\lstdefinestyle{mystyle}{
    language=C++,
    basicstyle=\ttfamily\footnotesize,
    backgroundcolor=\color{backcolour},
    commentstyle=\color{codegreen},
    keywordstyle=\color{magenta},
    numberstyle=\tiny\color{codegray},
    stringstyle=\color{codepurple},
    breakatwhitespace=false,
    breaklines=true,
    captionpos=b,
    keepspaces=true,
    numbers=left,
    numbersep=5pt,
    showspaces=false,
    showstringspaces=false,
    showtabs=false,
    tabsize=2
}

\title{Lab \# 10: Enumerations}
\subtitle{EC-102 -- Computer Systems and Programming}

\author{Usman Ayub Sheikh}
\institute{School of Mechanical and Manufacturing Engineering (SMME), \\ National University of Sciences and Technology (NUST)}
\date{\today}

\begin{document}
\begin{frame}
    \titlepage
\end{frame}

\begin{frame}
    \frametitle{Outline}
        \tableofcontents
\end{frame}

\begin{frame}
    \frametitle{Enumerations -- Introduction}
    \section{Enumerations} % (fold)
    \label{sec:enumerations}
    \subsection{Introduction} % (fold)
    \label{sub:introduction}
    \begin{itemize}
        \item Another approach to defining your own data type
        \item When we know in advance a \textbf{finite} list of values that a data type can take on
        \item Example: A data type representing days of the week
        \item In an enumeration, we must give a specific name to every possible value
        \item These permissible values of an enumeration are known as \textbf{enumerators}
    \end{itemize}
\end{frame}

\begin{frame}[fragile]
    \frametitle{Enumerations -- Syntax}
    \subsection{Syntax} % (fold)
    \label{sub:syntax}
    \begin{itemize}
        \item An \texttt{enum} declaration defines the set of all names that will be permissible values of the type
        \lstset{style=mystyle}
\begin{lstlisting}
enum days_of_the_week {Sun, Mon, Tue, Wed, Thu, Fri, Sat};
\end{lstlisting}
        \item In this example, the \texttt{enum} type \texttt{days\_of\_the\_week} has seven enumerators: Sun, Mon, Tue and so on up to Sat
        \item Once \texttt{enum} type \texttt{days\_of\_the\_week} has been declared as shown, the variables of this type can be defined as follows:
        \lstset{style=mystyle}
\begin{lstlisting}
days_of_the_week day1, day2;
\end{lstlisting}
        \item Variables of an enumerated type, like \texttt{day1} and \texttt{day2} can be given any of the values listed in the enum declaration
        \item Values that weren't listed in the declaration cannot be used
    \end{itemize}
\end{frame}

\begin{frame}[fragile]
    \frametitle{Enumerations -- Other Features}
    \subsection{Other Features} % (fold)
    \label{sub:other_features}
    \begin{itemize}
        \item Enumerations are treated internally as integers
        \item We can use standard arithmetic operators on standard \texttt{enum} types
        \lstset{style=mystyle}
\begin{lstlisting}
day1 = Mon;
day2 = Thu;

int diff = day2 - day1;

cout << "Days between = " << diff << endl;
\end{lstlisting}
        \item We can also use comparison operators
        \lstset{style=mystyle}
\begin{lstlisting}
if(day1 < day2)
{
    cout << day1 << " comes before " << day2 << endl;
}
\end{lstlisting}
    \end{itemize}
\end{frame}

\begin{frame} [fragile]
    \frametitle{Solved Example}
    \section{Solved Example} % (fold)
    \label{sec:solved_example}
    \lstset{style=mystyle}
\begin{lstlisting}
// demonstrates enum types
#include <iostream>
using namespace std;

enum days_of_the_week
{
    Sun = 1,
    Mon = 2,
    Tue = 3,
    Wed = 4,
    Thu = 6,
    Fri = 7,
    Sat = 8
};

int main()
{
    days_of_the_week day1, day2;
\end{lstlisting}
\end{frame}

\begin{frame} [fragile]
    \frametitle{Solved Example}
    \lstset{style=mystyle}
\begin{lstlisting} [firstnumber=19]
    day1 = Mon;
    day2 = Thu;

    int diff = day2 - day1; // integer arithmetic
    cout << "Days between = " << diff << endl;
    if (day1 < day2) // can do comparisons
    {
        cout << "day1 comes before day2" << endl;
    }
    return 0;
}
\end{lstlisting}
\end{frame}

\end{document}