\documentclass{beamer}
\usepackage[utf8]{inputenc}

\usepackage{amsmath}
\usepackage{graphicx}
\usepackage{url}
\usepackage{fancyvrb}
\usepackage{xcolor}
\usepackage{adjustbox}

\usetheme{Madrid}
\usecolortheme{seahorse}

\usepackage{inconsolata}
\usepackage[scaled]{helvet}
\renewcommand*\familydefault{\sfdefault}
\usepackage[T1]{fontenc}

\usepackage{listings}
\usepackage{color}

\definecolor{codegreen}{rgb}{0,0.6,0}
\definecolor{codegray}{rgb}{0.5,0.5,0.5}
\definecolor{codepurple}{rgb}{0.58,0,0.82}
\definecolor{backcolour}{rgb}{0.95,0.95,0.92}

\mode<presentation>

\definecolor{orange}{HTML}{BC2E07}

\usepackage{hyperref}
\hypersetup{
    colorlinks,
    linkcolor=orange,
    urlcolor=blue
}

\lstdefinestyle{mystyle}{
    language=C++,
    basicstyle=\ttfamily\footnotesize,
    backgroundcolor=\color{backcolour},
    commentstyle=\color{codegreen},
    keywordstyle=\color{magenta},
    numberstyle=\tiny\color{codegray},
    stringstyle=\color{codepurple},
    breakatwhitespace=false,
    breaklines=true,
    captionpos=b,
    keepspaces=true,
    numbers=left,
    numbersep=5pt,
    showspaces=false,
    showstringspaces=false,
    showtabs=false,
    tabsize=2
}

\usepackage{datetime}
\newdate{date}{14}{12}{2015}

\title{Lab \# 15: Arrays and Strings -- Part 1}
\subtitle{EC-102 -- Computer Systems and Programming}

\author{Usman Ayub Sheikh}
\institute{School of Mechanical and Manufacturing Engineering (SMME), \\ National University of Sciences and Technology (NUST)}
\date{\displaydate{date}}

\begin{document}

\begin{frame}
    \titlepage
\end{frame}

\begin{frame}
    \frametitle{Outline}
        \tableofcontents
\end{frame}

\begin{frame}\frametitle{Why Array?}
    \section{Arrays} % (fold)
    \label{sec:arrays}
    \subsection{Why?} % (fold)
    \label{sub:why_}
    \begin{itemize}
        \item In everyday life, we commonly \textbf{group} similar objects into units
        \begin{itemize}
            \item A crate of peas, and
            \item A dozen of eggs
         \end{itemize}
        \item In computer programs, we also need to group together data items \textbf{of the same type}
        \item The most basic mechanism that accomplishes this in C++ is the \textbf{array}
        \item The data items grouped in an array can be simple types such \texttt{int}, \texttt{float} or they can be user-defined types such as structures
    \end{itemize}
    % subsection why (end)
    % section arrays (end)
\end{frame}

\begin{frame}\frametitle{What is an Array?}
    \subsection{What?} % (fold)
    \label{sub:what_}
    \begin{itemize}
        \item Like structures, arrays also group a number of items into a larger unit
        \item But while structures usually group items of different types, an array groups items of the same type
        \item Whereas items in a structure are accessed by name, those in an array \textbf{are accessed by an index number}
    \end{itemize}
    % subsection what_ (end)
\end{frame}

\begin{frame}[fragile]\frametitle{A Simple Program Using Array}
    \subsection{How?} % (fold)
    \label{sub:how_}
    \lstset{style=mystyle}
\begin{lstlisting}
// gets four ages from user and displays them
#include <iostream>
using namespace std;

int main()
{
    int age[4];
    for(int j = 0; j < 4; j++)
    {
        cout << "Enter an age: ";
        cin >> age[j];
    }

    for(int j = 0; j < 4; j++)
    {
        cout << "You entered: " << age[j] << endl;
    }
    return 0;
}
\end{lstlisting}
    % subsection how_ (end)
\end{frame}

\begin{frame}[fragile]\frametitle{A Simple Program Using Array}
    \subsubsection{Defining Array} % (fold)
    \label{ssub:defining_array}
    \textbf{Defining Array:} \\
    \begin{itemize}
        \item An array must be defined before it can be used
        \item An array definition specifies \textbf{a variable type} and \textbf{a name}
        \item Unlike variable definitions, it includes another feature: \textbf{a size}
        \begin{center}
            \texttt{int age[4];}
        \end{center}
    \end{itemize}
    % subsubsection defining_arrays (end)

    \subsubsection{Array Elements} % (fold)
    \label{ssub:array_elements}
    \textbf{Array Elements:} \\
    \begin{itemize}
        \item The items in an array
        \item All the elements in an array are of the \textbf{same type}
        \item The first element is numbered \texttt{0}
    \end{itemize}
    % subsubsection array_elements (end)

    \subsubsection{Accessing Array Elements} % (fold)
    \label{ssub:accessing_array_elements}
    \textbf{Accessing Array Elements:} \\
    \begin{itemize}
         \item Name of the array followed by brackets delimiting a variable or a constant known as the \textbf{array index}
         \item \texttt{age[j]} refers to the $j^{th}$ element of the array \texttt{age}
     \end{itemize}
    % subsubsection accessing_array_elements (end)
\end{frame}

\begin{frame}[fragile]\frametitle{Averaging Array Elements}
    \section{Solved Examples} % (fold)
    \label{sec:solved_examples}
    \subsection{Averaging Array Elements} % (fold)
    \label{sub:averaging_array_elements}
    \lstset{style=mystyle}
\begin{lstlisting}
// averages a weeks's sales
#include <iostream>
using namespace std;

int main()
{
    const int SIZE = 6;
    double sales[SIZE];

    cout << "Enter widget sales for 6 days:\n";
    for (int j = 0; j < SIZE; j++)
        cin >> sales[j];

    double total = 0;
    for (int j = 0; j < SIZE; j++)
        total += sales[j];

    cout << "Average: " << total / SIZE << endl;
    return 0;
}
\end{lstlisting}
    % subsection averaging_array_elements (end)
    % section solved_examples (end)
\end{frame}

\begin{frame}\frametitle{Averaging Array Elements}
    \begin{itemize}
        \item Using a variable (\texttt{SIZE}) makes it easier to change the array size
        \item Only one program line needs to be changed to change the array size, loop limits and anywhere else the array size appears
        \item Using all upper-case name reminds us that the variable cannot be modified in the program
    \end{itemize}
\end{frame}

\begin{frame}[fragile]\frametitle{Initializing Arrays}
    \subsection{Initializing Arrays} % (fold)
    \label{sub:initializing_arrays}
    \lstset{style=mystyle}
\begin{lstlisting}
// shows days from start of the year to date specified
#include <iostream>
using namespace std;
int main()
{
    int month, day, total_days;
    int days_per_month[12] = {31, 28, 31, 30, 31, 30, 31, 31, 30, 31, 30, 31};

    cout << "Enter month (1 to 12): ";
    cin >> month;
    cout << "Enter day (1 to 31): ";
    cin >> day;
    total_days = day;
    for (int j = 0; j < month - 1; j++)
        total_days += days_per_month[j];
    cout << "Total days from start of the year is: " << total_days << endl;
    return 0;
}
\end{lstlisting}
    % subsection initializing_arrays (end)
\end{frame}

\begin{frame}\frametitle{Initializing Arrays}
    \begin{itemize}
        \item The values to which an array is initialized \textbf{are surrounded by braces} and \textbf{are separated by commas}
        \begin{center}
            \texttt{int days\_per\_month[12] = \{31, 28, 31, 30, 31, 30, 31, 31, 30, 31, 30, 31\};}
        \end{center}
        \item What happens if you do use an explicit array size but it does not agree with the number of initializers?
        \begin{itemize}
            \item If there are too few initializers, \textbf{the missing elements will be set to \texttt{0}}
            \begin{center}
                \texttt{int days\_per\_month[12] = \{31, 28, 31, 30, 31, 30, 31\};}
            \end{center}
            In this case, \texttt{days\_per\_month[7]} to \texttt{[11]} will all be set to \texttt{0}
            \item If there are too many, \textbf{an error is signaled}
            \begin{center}
                \texttt{int days\_per\_month[12] = \{31, 28, 31, 30, 31, 30, 31, 31, 30, 31, 30, 31, {\color{red}{30, 31, 30, 31}}\};}
            \end{center}
        \end{itemize}
    \end{itemize}
\end{frame}

\begin{frame}[fragile]\frametitle{Multi-dimensional Arrays}
    \subsection{Multi-dimensional Arrays} % (fold)
    \label{sub:multi_dimensional_arrays}
    \lstset{style=mystyle}
\begin{lstlisting}
// displays sales chart using 2-d array
#include <iostream>
#include <iomanip>
using namespace std;

const int DISTRICTS = 4; // array dimensions
const int MONTHS = 3;

int main()
{
    int d, m;
    double sales[DISTRICTS][MONTHS]; // 2-d array defintion
    cout << endl;
    for (d = 0; d < DISTRICTS; d++)
        for (m = 0; m < MONTHS; m++)
        {
            cout << "Enter sales for district " << d + 1;
            cout << ", month " << m + 1 << ": ";
            cin >> sales[d][m];
        }
\end{lstlisting}
    % subsection multi_dimensional_arrays (end)
\end{frame}

\begin{frame}[fragile]\frametitle{Multi-dimensional Arrays}
    \lstset{style=mystyle}
\begin{lstlisting} [firstnumber=21]
    cout << "\n\n";
    cout << "                        Month\n";
    cout << "                1         2         3";
    for (d = 0; d < DISTRICTS; d++)
    {
        cout << "\nDistrict " << d + 1;
        for (m = 0; m < MONTHS; m++)
        {
            cout << setw(10) << sales[d][m];
        }
    }
    cout << endl;
    return 0;
}
\end{lstlisting}
    % subsection multi_dimensional_arrays (end)
\end{frame}

\begin{frame}[fragile]\frametitle{Multi-dimensional Arrays}
    \subsubsection{Defining Multi-dimensional Array} % (fold)
    \label{ssub:defining_multi_dimensional_arrays}
    \textbf{Defining Multi-dimensional Arrays:} \\
    \begin{itemize}
        \item It is an array of \texttt{DISTRICTS} elements, each of which is an array of \texttt{MONTHS} elements
        \begin{center}
            \texttt{double sales[DISTRICTS][MONTHS];}
        \end{center}
        \item There can be arrays of more than two dimensions e.g. a three dimensional array is an array of arrays of arrays
    \end{itemize}
    % subsubsection defining_multi_dimensional_arrays (end)

    \subsubsection{Accessing Multi-dimensional Array Elements} % (fold)
    \label{ssub:accessing_multi_dimensional_array_elements}
    \textbf{Accessing Array Elements:} \\
    \begin{itemize}
        \item Each index has its own set of brackets
        \begin{center}
            \texttt{sales[d][m]}
        \end{center}
        \item Commas are not used
     \end{itemize}
    % subsubsection accessing_multi_dimensional_array_elements (end)
\end{frame}

\begin{frame}[fragile]\frametitle{Multi-dimensional Arrays}
    \subsubsection{Initializing Multi-dimensional Arrays} % (fold)
    \label{ssub:initializing_multi_dimensional_arrays}
    \textbf{Initializing Multi-dimensional Arrays:} \\
    The initializing values for each sub-array are enclosed in braces and separated by commas
    \lstset{style=mystyle}
            \begin{lstlisting}
    double sales[DISTRICTS][MONTHS] = {
        {1432.07, 234.50, 654.01},
        {322.00, 13838.32, 17589.88},
        {9328.34, 934.00, 4492.30},
        {12838.29, 2332.63, 32.93}
    };
\end{lstlisting}
    % subsubsection initializing_multi_dimensional_arrays (end)
\end{frame}

\begin{frame}\frametitle{Exercise 1}
\section{Exercise} % (fold)
\label{sec:exercise}
\subsection{Exercise 1} % (fold)
\label{sub:exercise_1}
Write a program that:
\begin{itemize}
    \item Declares an array containing ten elements from 1 to 10
    \item Multiplies the number entered by the user with this array, and
    \item Displays the output
\end{itemize}
% subsection exercise_1 (end)
\end{frame}

\begin{frame}\frametitle{Exercise 2}
\subsection{Exercise 2} % (fold)
\label{sub:exercise_2}
Write a program that:
\begin{itemize}
    \item Asks the user to enter 5 numbers
    \item Writes those numbers to an array
    \item Arranges the elements of the array in ascending order, and
    \item Displays the array
\end{itemize}
% subsection exercise_1 (end)
\end{frame}

\end{document}